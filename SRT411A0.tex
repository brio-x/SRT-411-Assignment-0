\documentclass[]{article}
\usepackage{lmodern}
\usepackage{amssymb,amsmath}
\usepackage{ifxetex,ifluatex}
\usepackage{fixltx2e} % provides \textsubscript
\ifnum 0\ifxetex 1\fi\ifluatex 1\fi=0 % if pdftex
  \usepackage[T1]{fontenc}
  \usepackage[utf8]{inputenc}
\else % if luatex or xelatex
  \ifxetex
    \usepackage{mathspec}
  \else
    \usepackage{fontspec}
  \fi
  \defaultfontfeatures{Ligatures=TeX,Scale=MatchLowercase}
\fi
% use upquote if available, for straight quotes in verbatim environments
\IfFileExists{upquote.sty}{\usepackage{upquote}}{}
% use microtype if available
\IfFileExists{microtype.sty}{%
\usepackage{microtype}
\UseMicrotypeSet[protrusion]{basicmath} % disable protrusion for tt fonts
}{}
\usepackage[margin=1in]{geometry}
\usepackage{hyperref}
\hypersetup{unicode=true,
            pdftitle={SRT411A0},
            pdfauthor={Katelin Stuart},
            pdfborder={0 0 0},
            breaklinks=true}
\urlstyle{same}  % don't use monospace font for urls
\usepackage{color}
\usepackage{fancyvrb}
\newcommand{\VerbBar}{|}
\newcommand{\VERB}{\Verb[commandchars=\\\{\}]}
\DefineVerbatimEnvironment{Highlighting}{Verbatim}{commandchars=\\\{\}}
% Add ',fontsize=\small' for more characters per line
\usepackage{framed}
\definecolor{shadecolor}{RGB}{248,248,248}
\newenvironment{Shaded}{\begin{snugshade}}{\end{snugshade}}
\newcommand{\KeywordTok}[1]{\textcolor[rgb]{0.13,0.29,0.53}{\textbf{#1}}}
\newcommand{\DataTypeTok}[1]{\textcolor[rgb]{0.13,0.29,0.53}{#1}}
\newcommand{\DecValTok}[1]{\textcolor[rgb]{0.00,0.00,0.81}{#1}}
\newcommand{\BaseNTok}[1]{\textcolor[rgb]{0.00,0.00,0.81}{#1}}
\newcommand{\FloatTok}[1]{\textcolor[rgb]{0.00,0.00,0.81}{#1}}
\newcommand{\ConstantTok}[1]{\textcolor[rgb]{0.00,0.00,0.00}{#1}}
\newcommand{\CharTok}[1]{\textcolor[rgb]{0.31,0.60,0.02}{#1}}
\newcommand{\SpecialCharTok}[1]{\textcolor[rgb]{0.00,0.00,0.00}{#1}}
\newcommand{\StringTok}[1]{\textcolor[rgb]{0.31,0.60,0.02}{#1}}
\newcommand{\VerbatimStringTok}[1]{\textcolor[rgb]{0.31,0.60,0.02}{#1}}
\newcommand{\SpecialStringTok}[1]{\textcolor[rgb]{0.31,0.60,0.02}{#1}}
\newcommand{\ImportTok}[1]{#1}
\newcommand{\CommentTok}[1]{\textcolor[rgb]{0.56,0.35,0.01}{\textit{#1}}}
\newcommand{\DocumentationTok}[1]{\textcolor[rgb]{0.56,0.35,0.01}{\textbf{\textit{#1}}}}
\newcommand{\AnnotationTok}[1]{\textcolor[rgb]{0.56,0.35,0.01}{\textbf{\textit{#1}}}}
\newcommand{\CommentVarTok}[1]{\textcolor[rgb]{0.56,0.35,0.01}{\textbf{\textit{#1}}}}
\newcommand{\OtherTok}[1]{\textcolor[rgb]{0.56,0.35,0.01}{#1}}
\newcommand{\FunctionTok}[1]{\textcolor[rgb]{0.00,0.00,0.00}{#1}}
\newcommand{\VariableTok}[1]{\textcolor[rgb]{0.00,0.00,0.00}{#1}}
\newcommand{\ControlFlowTok}[1]{\textcolor[rgb]{0.13,0.29,0.53}{\textbf{#1}}}
\newcommand{\OperatorTok}[1]{\textcolor[rgb]{0.81,0.36,0.00}{\textbf{#1}}}
\newcommand{\BuiltInTok}[1]{#1}
\newcommand{\ExtensionTok}[1]{#1}
\newcommand{\PreprocessorTok}[1]{\textcolor[rgb]{0.56,0.35,0.01}{\textit{#1}}}
\newcommand{\AttributeTok}[1]{\textcolor[rgb]{0.77,0.63,0.00}{#1}}
\newcommand{\RegionMarkerTok}[1]{#1}
\newcommand{\InformationTok}[1]{\textcolor[rgb]{0.56,0.35,0.01}{\textbf{\textit{#1}}}}
\newcommand{\WarningTok}[1]{\textcolor[rgb]{0.56,0.35,0.01}{\textbf{\textit{#1}}}}
\newcommand{\AlertTok}[1]{\textcolor[rgb]{0.94,0.16,0.16}{#1}}
\newcommand{\ErrorTok}[1]{\textcolor[rgb]{0.64,0.00,0.00}{\textbf{#1}}}
\newcommand{\NormalTok}[1]{#1}
\usepackage{graphicx,grffile}
\makeatletter
\def\maxwidth{\ifdim\Gin@nat@width>\linewidth\linewidth\else\Gin@nat@width\fi}
\def\maxheight{\ifdim\Gin@nat@height>\textheight\textheight\else\Gin@nat@height\fi}
\makeatother
% Scale images if necessary, so that they will not overflow the page
% margins by default, and it is still possible to overwrite the defaults
% using explicit options in \includegraphics[width, height, ...]{}
\setkeys{Gin}{width=\maxwidth,height=\maxheight,keepaspectratio}
\IfFileExists{parskip.sty}{%
\usepackage{parskip}
}{% else
\setlength{\parindent}{0pt}
\setlength{\parskip}{6pt plus 2pt minus 1pt}
}
\setlength{\emergencystretch}{3em}  % prevent overfull lines
\providecommand{\tightlist}{%
  \setlength{\itemsep}{0pt}\setlength{\parskip}{0pt}}
\setcounter{secnumdepth}{0}
% Redefines (sub)paragraphs to behave more like sections
\ifx\paragraph\undefined\else
\let\oldparagraph\paragraph
\renewcommand{\paragraph}[1]{\oldparagraph{#1}\mbox{}}
\fi
\ifx\subparagraph\undefined\else
\let\oldsubparagraph\subparagraph
\renewcommand{\subparagraph}[1]{\oldsubparagraph{#1}\mbox{}}
\fi

%%% Use protect on footnotes to avoid problems with footnotes in titles
\let\rmarkdownfootnote\footnote%
\def\footnote{\protect\rmarkdownfootnote}

%%% Change title format to be more compact
\usepackage{titling}

% Create subtitle command for use in maketitle
\newcommand{\subtitle}[1]{
  \posttitle{
    \begin{center}\large#1\end{center}
    }
}

\setlength{\droptitle}{-2em}

  \title{SRT411A0}
    \pretitle{\vspace{\droptitle}\centering\huge}
  \posttitle{\par}
    \author{Katelin Stuart}
    \preauthor{\centering\large\emph}
  \postauthor{\par}
      \predate{\centering\large\emph}
  \postdate{\par}
    \date{February 15, 2019}


\begin{document}
\maketitle

\begin{enumerate}
\def\labelenumi{\arabic{enumi}.}
\tightlist
\item
  Compute the difference between 2014 and the year you started at this
  university and divide this by the difference between 2014 and the year
  you were born. Multiply this with 100 to get the percentage of your
  life you have spent at this university. Use brackets if you need them.
\end{enumerate}

\begin{Shaded}
\begin{Highlighting}[]
\NormalTok{((}\DecValTok{2017}\OperatorTok{-}\DecValTok{2014}\NormalTok{)}\OperatorTok{/}\NormalTok{(}\DecValTok{2014}\OperatorTok{-}\DecValTok{1987}\NormalTok{))}\OperatorTok{*}\DecValTok{100}
\end{Highlighting}
\end{Shaded}

\begin{verbatim}
## [1] 11.11111
\end{verbatim}

2.Repeat the previous ToDo, but with several steps in between. You can
give the variables any name you want, but the name has to start with a
letter.

\begin{Shaded}
\begin{Highlighting}[]
\NormalTok{start_year=}\DecValTok{2017}
\NormalTok{base_year=}\DecValTok{2014}
\NormalTok{bday=}\DecValTok{1987}
\NormalTok{percentage=}\DecValTok{100}
\NormalTok{((start_year}\OperatorTok{-}\NormalTok{base_year)}\OperatorTok{/}\NormalTok{(base_year}\OperatorTok{-}\NormalTok{bday))}\OperatorTok{*}\NormalTok{percentage}
\end{Highlighting}
\end{Shaded}

\begin{verbatim}
## [1] 11.11111
\end{verbatim}

\begin{enumerate}
\def\labelenumi{\arabic{enumi}.}
\setcounter{enumi}{2}
\tightlist
\item
  Compute the sum of 4,5,8 and 11 by first combing them into a vector
  and then using the function sum.
\end{enumerate}

\begin{Shaded}
\begin{Highlighting}[]
\NormalTok{k=}\KeywordTok{c}\NormalTok{(}\DecValTok{4}\NormalTok{,}\DecValTok{5}\NormalTok{,}\DecValTok{8}\NormalTok{,}\DecValTok{11}\NormalTok{) }
\KeywordTok{sum}\NormalTok{(k)}
\end{Highlighting}
\end{Shaded}

\begin{verbatim}
## [1] 28
\end{verbatim}

\begin{enumerate}
\def\labelenumi{\arabic{enumi}.}
\setcounter{enumi}{3}
\tightlist
\item
  Plot 100 normal random numbers.
\end{enumerate}

\begin{Shaded}
\begin{Highlighting}[]
\NormalTok{x =}\StringTok{ }\KeywordTok{rnorm}\NormalTok{(}\DecValTok{100}\NormalTok{)}
\KeywordTok{plot}\NormalTok{(x)}
\end{Highlighting}
\end{Shaded}

\includegraphics{SRT411A0_files/figure-latex/unnamed-chunk-4-1.pdf} 5.
Find help for the sqrt function

\begin{Shaded}
\begin{Highlighting}[]
\KeywordTok{help}\NormalTok{(sqrt)}
\end{Highlighting}
\end{Shaded}

\begin{enumerate}
\def\labelenumi{\arabic{enumi}.}
\setcounter{enumi}{5}
\tightlist
\item
  Make a file called firstscript.R containing R-code that generates 100
  random numbes and plots them, and run this script several times.
\end{enumerate}

\begin{Shaded}
\begin{Highlighting}[]
\KeywordTok{setwd}\NormalTok{(}\StringTok{"/home/kaitee/assignments/0"}\NormalTok{)}
\KeywordTok{source}\NormalTok{(}\StringTok{"firstscript.r"}\NormalTok{)}
\end{Highlighting}
\end{Shaded}

\includegraphics{SRT411A0_files/figure-latex/unnamed-chunk-6-1.pdf} 7.
Put the numbers 31 to 60 in a vector named P and in a matrix with 6 rows
and 5 columns named Q. Tip: use the function seq. Look at the different
ways scalars, vectors and matrices are denoted in the workspace window.

\begin{Shaded}
\begin{Highlighting}[]
\NormalTok{P=}\KeywordTok{seq}\NormalTok{(}\DecValTok{31}\NormalTok{,}\DecValTok{60}\NormalTok{)}
\NormalTok{Q=}\KeywordTok{matrix}\NormalTok{(P,}\DataTypeTok{ncol=}\DecValTok{5}\NormalTok{,}\DataTypeTok{nrow=}\DecValTok{6}\NormalTok{)}
\NormalTok{Q}
\end{Highlighting}
\end{Shaded}

\begin{verbatim}
##      [,1] [,2] [,3] [,4] [,5]
## [1,]   31   37   43   49   55
## [2,]   32   38   44   50   56
## [3,]   33   39   45   51   57
## [4,]   34   40   46   52   58
## [5,]   35   41   47   53   59
## [6,]   36   42   48   54   60
\end{verbatim}

\begin{enumerate}
\def\labelenumi{\arabic{enumi}.}
\setcounter{enumi}{7}
\tightlist
\item
  Make a script file which constructs three random normal vectors of
  length 100. Call these vectors x1, x2 and x3. Make a data frame called
  t with three columns (called a, b and c) containing respectively x1,
  x1+x2 and x1+x2+x3.Call the following functions for this data
  frame:plot(t) and sd(t). Can you understand the results? Rerun this
  script a few times.
\end{enumerate}

\begin{Shaded}
\begin{Highlighting}[]
\KeywordTok{source}\NormalTok{(}\StringTok{"script_vectors.r"}\NormalTok{)}
\end{Highlighting}
\end{Shaded}

\includegraphics{SRT411A0_files/figure-latex/unnamed-chunk-8-1.pdf}
\includegraphics{SRT411A0_files/figure-latex/unnamed-chunk-8-2.pdf} 9.
Add these lines to the script file of the previous section. Try to find
out, either by experimenting or by using the help, what the meaning is
of rgb, the last argument of rgb, lwd, pch, cex.

\begin{Shaded}
\begin{Highlighting}[]
\KeywordTok{source}\NormalTok{(}\StringTok{"script_vectors.r"}\NormalTok{)}
\end{Highlighting}
\end{Shaded}

\includegraphics{SRT411A0_files/figure-latex/unnamed-chunk-9-1.pdf}
\includegraphics{SRT411A0_files/figure-latex/unnamed-chunk-9-2.pdf} 10.
Make a file called tst1.txt in Notepad from the example in Figure 4 and
store it in your working directory. Write a script to read it, to
multiply the column called g by 5 and to store it as tst2.txt.

\begin{Shaded}
\begin{Highlighting}[]
\NormalTok{my_var=}\KeywordTok{read.table}\NormalTok{(}\DataTypeTok{file=}\StringTok{"tst1.txt"}\NormalTok{,}\DataTypeTok{header=}\OtherTok{TRUE}\NormalTok{)}
\NormalTok{my_var2=my_var}\OperatorTok{$}\NormalTok{g}\OperatorTok{*}\DecValTok{5}
\KeywordTok{write.table}\NormalTok{(my_var2, }\DataTypeTok{file=}\StringTok{"tst2.txt"}\NormalTok{, }\DataTypeTok{row.names=}\OtherTok{FALSE}\NormalTok{)}
\KeywordTok{read.table}\NormalTok{(}\DataTypeTok{file=}\StringTok{"tst2.txt"}\NormalTok{, }\DataTypeTok{header=}\OtherTok{TRUE}\NormalTok{)}
\end{Highlighting}
\end{Shaded}

\begin{verbatim}
##     x
## 1  10
## 2  20
## 3  40
## 4  80
## 5 160
## 6 320
\end{verbatim}

\begin{enumerate}
\def\labelenumi{\arabic{enumi}.}
\setcounter{enumi}{10}
\tightlist
\item
  Compute the mean of the square root of a vector of 100 random numbers.
  What happens?
\end{enumerate}

\begin{Shaded}
\begin{Highlighting}[]
\KeywordTok{mean}\NormalTok{(}\KeywordTok{sqrt}\NormalTok{(}\KeywordTok{rnorm}\NormalTok{(}\DecValTok{100}\NormalTok{)))}
\end{Highlighting}
\end{Shaded}

\begin{verbatim}
## Warning in sqrt(rnorm(100)): NaNs produced
\end{verbatim}

\begin{verbatim}
## [1] NaN
\end{verbatim}

\begin{enumerate}
\def\labelenumi{\arabic{enumi}.}
\setcounter{enumi}{11}
\tightlist
\item
  Make a graph with on the x-axis: today, Sinterklaas 2014 and your next
  birthday and oncthe y-axis the number of presents you expect oneach of
  these days. Tip: make two vectors first.
\end{enumerate}

\begin{Shaded}
\begin{Highlighting}[]
\NormalTok{date=}\KeywordTok{strptime}\NormalTok{( }\KeywordTok{c}\NormalTok{(}\DecValTok{20190215}\NormalTok{, }\DecValTok{20141225}\NormalTok{, }\DecValTok{20191009}\NormalTok{), }\DataTypeTok{format=}\StringTok{"%Y%m%d"}\NormalTok{)}
\NormalTok{presents=}\KeywordTok{c}\NormalTok{(}\DecValTok{0}\NormalTok{,}\DecValTok{5}\NormalTok{,}\DecValTok{1}\NormalTok{)}
\NormalTok{t=}\KeywordTok{data.frame}\NormalTok{(}\DataTypeTok{x=}\NormalTok{date, }\DataTypeTok{y=}\NormalTok{presents)}
\KeywordTok{plot}\NormalTok{(date, presents, }\DataTypeTok{col=}\StringTok{"dark green"}\NormalTok{)}
\end{Highlighting}
\end{Shaded}

\includegraphics{SRT411A0_files/figure-latex/unnamed-chunk-12-1.pdf} 13.
Make a vector from 1 to 100. Make a for-loop which runs through the
whole vector. Multiply the elements which are smaller than 5 and larger
than 90 with 10 and the other elements with 0.1.

\begin{Shaded}
\begin{Highlighting}[]
\NormalTok{q=}\KeywordTok{seq}\NormalTok{(}\DataTypeTok{from=}\DecValTok{1}\NormalTok{, }\DataTypeTok{to=}\DecValTok{100}\NormalTok{)}
\NormalTok{s=}\KeywordTok{c}\NormalTok{()}
 \ControlFlowTok{for}\NormalTok{(i }\ControlFlowTok{in} \DecValTok{1}\OperatorTok{:}\DecValTok{100}\NormalTok{)}
\NormalTok{ \{}
   \ControlFlowTok{if}\NormalTok{(q[i]}\OperatorTok{<}\DecValTok{5}\NormalTok{)}
\NormalTok{   \{}
\NormalTok{     s[i]=q[i]}\OperatorTok{*}\DecValTok{5}
\NormalTok{   \}}
   \ControlFlowTok{else} \ControlFlowTok{if}\NormalTok{ (q[i]}\OperatorTok{>}\DecValTok{90}\NormalTok{)}
\NormalTok{   \{}
\NormalTok{     s[i]=q[i]}\OperatorTok{*}\DecValTok{10}
\NormalTok{   \}}
   \ControlFlowTok{else}
\NormalTok{   \{}
\NormalTok{     s[i]=q[i]}\OperatorTok{*}\FloatTok{0.1}
\NormalTok{   \} }
 
\NormalTok{ \}}
\NormalTok{s}
\end{Highlighting}
\end{Shaded}

\begin{verbatim}
##   [1]    5.0   10.0   15.0   20.0    0.5    0.6    0.7    0.8    0.9    1.0
##  [11]    1.1    1.2    1.3    1.4    1.5    1.6    1.7    1.8    1.9    2.0
##  [21]    2.1    2.2    2.3    2.4    2.5    2.6    2.7    2.8    2.9    3.0
##  [31]    3.1    3.2    3.3    3.4    3.5    3.6    3.7    3.8    3.9    4.0
##  [41]    4.1    4.2    4.3    4.4    4.5    4.6    4.7    4.8    4.9    5.0
##  [51]    5.1    5.2    5.3    5.4    5.5    5.6    5.7    5.8    5.9    6.0
##  [61]    6.1    6.2    6.3    6.4    6.5    6.6    6.7    6.8    6.9    7.0
##  [71]    7.1    7.2    7.3    7.4    7.5    7.6    7.7    7.8    7.9    8.0
##  [81]    8.1    8.2    8.3    8.4    8.5    8.6    8.7    8.8    8.9    9.0
##  [91]  910.0  920.0  930.0  940.0  950.0  960.0  970.0  980.0  990.0 1000.0
\end{verbatim}

\begin{enumerate}
\def\labelenumi{\arabic{enumi}.}
\setcounter{enumi}{13}
\tightlist
\item
  Write a function for the previous ToDo, so that you can feed it any
  vector you like (as argument). Use a for-loop in the function to do
  the computation with each element. Use the standard R function length
  in the specification of the counter.
\end{enumerate}

\begin{Shaded}
\begin{Highlighting}[]
\NormalTok{my_func=}\ControlFlowTok{function}\NormalTok{(arg1, arg2)}
\NormalTok{\{}
\NormalTok{  q[i]=arg1[i];}
  \ControlFlowTok{for}\NormalTok{ (i }\ControlFlowTok{in} \KeywordTok{length}\NormalTok{(q))}
\NormalTok{  \{}
    
\NormalTok{  \}}
\NormalTok{\}}
\end{Highlighting}
\end{Shaded}


\end{document}
